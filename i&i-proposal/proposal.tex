\documentclass[]{article}


\author{Silvio Gregorini}
\title{}
\date{2019}

% Ask if it's possible to use a two column layout
\usepackage[utf8]{inputenc}
\usepackage{csquotes}
\usepackage[english]{babel}
\usepackage{xcolor}
\usepackage[urlbordercolor=white, linkbordercolor=white]{hyperref}
%\usepackage[style=authoryear]{biblatex}
%\addbibresource{bibliography.bib}

\newcommand{\bibTitle}[1]{\emph{#1}}
\newcommand{\bibYear}[1]{(#1)}
\newcommand{\bibUrl}[1]{\textless#1\textgreater}
\newcommand{\bibUrldate}[1]{[Accessed #1]}

\newcommand{\citeInText}[2]{#1 (#2)}

% Change appearance of numeric labels in citation call-outs
\usepackage{cite}
\renewcommand\citeleft{(}
\renewcommand\citeright{)}

% Change appearance of numeric labels in bibliography 
\makeatletter
\renewcommand{\@biblabel}[1]{}
\makeatother

%\renewcommand{\urlbordercolor}{0} {0} {0}

\begin{document}

    In this module, you will develop an original interactive or reactive system
 in which music or sound is a key component. Innovative and imaginative
  methods of interaction by a participant should be explored, and this
   could be for any relevant context (performance, composition, installation,
    game, sound toy, etc).
    
During the first half of the module, you should consider and develop a 
written proposal and plan for this project. Your proposal should be around 
2000 words and address the following areas:


%================================================================================
%PUT NUMBERS ON PARAGRAPHS

\section{Introduction} %DONE

    The aim of this paper is to propose the design and production of an 
    hardware synthesizer, starting from an existing digital sound engine.
    In the sections that follow, it will be explained the background and the rationale 
    behind the project, as well as the initial research undertaken.
    Then, it will be depicted a "blueprint" of the product: the general concept, the 
    architecture of the system -- both hardware and software -- and the production plan 
    will be covered in detail.
    In the last part the evaluation criteria will be set, in order to have a concrete measure
    of the work outcomes.

\section{Background and Motivation}

    This project has been shaped and will be realised keeping as a pivotal point the
    \emph{chiptune} subculture and its principles. As will be discussed in the next paragraphs,
    the product is conceived to be used by people already familiar with the environment and 
    limitations of this musical style. Its purpose is to give users a different and more 
    modern way to interact with a well-known set of sounds and synthesis capabilities.

    \subsection{Pushing the Limits Using Contraints} %WIP
        \paragraph{Chiptune}
        As stated by Collins et al. (2014 )\nocite{COLLINS2014} the term \emph{chiptune} has multiple 
        definitions. Also known as \emph{chip music} or \emph{8-bit music}, it derives from the sound chips that,
         in the first generation of computers and gaming consoles,
         were used to balance the processing power of generating sound effects and music from the
        CPU. In its strictiest meaning, chiptune is used to refer to music created entirely from the original, vintage audio 
        chips. Nevertheless, modifications that do not alter the nature of the sound produced are allowed.\\
        The broadest definition is more related to the aesthetics of the sound, rather then to the source
        generating it. The entire subculture which gravitates around the foundations and features of chip
        music can be called \emph{chiptune} too. Anyway, this paper and the related project will try to stick
        to the strictiest definition of the word, in order to create a product able to maintain the sound fidelity
        of the old processors.

        \paragraph{DMG-001 and trackers}
        In particular, the following study focuses on the Nintendo DMG-001 from 1989, known with the commercial 
        name of \emph{Game Boy}, which is supposedly one of the most popular tools for the production of chip music.
        From the official datasheet (Nintendo, 2019)\nocite{NINTENDO2019} portable console runs on a custom 
        \emph{Sharp LR35902} 8-bit CPU, similar to the \emph{Zilog Z80}\footnote{Popular 8-bit microprocessor widely used 
        from the 1970s to the mid-1980s in desktop and home computers, military applications, synthesizers, arcade machines\ldots} 
        and has four audio channels\footnote{To be precise, as stated in the 8BC Chiptune Wiki (2007)\nocite{8BCCHIPTUNE2007}, 
        the console has a fifth -- and least known -- channel: it is an analogue input channel that allows 
        external synthesis on cartridge to be mixed with the sound generated by the other channels. No cartridges 
        are known to use this channel and its functionality, though.}:\\[10pt]
        
        \def\arraystretch{1.2}
        \begin{tabular}{l l l}
            \hline
            \textbf{Channel} & \textbf{Type} & \textbf{Features}\\
            \hline \\[-6pt]
            1 & Quadrangular\footnotemark & - Volume envelope\\
                &   &                                                                          - 4-mode pulse width\\
                &   &                                                                          - Frequency register from C3 upwards\\
                &   &                                                                          - Frequency envelope\\            
            \hline
            2 & Quadrangular & - Volume envelope\\       
            &   &              - 4-mode pulse width\\
            &   &              - Frequency register from C3 upwards\\
            \hline
            3 & Wave & - User-definable waveforms\\
            &   &      - Bank of 32 samples (4-bit each)\\
            &   &      - Frequency register from C2 upwards\\
            \hline
            4 & Pseudo-random noise & - White and brown noise\\
            \hline
        \end{tabular}
        \footnotetext{Also known as \emph{pulse wave} or \emph{square wave}.}
        
        \nocite{MARQUEZ2014}


    \subsection{Rationale} %name


\section{Name of the Project} %name
    \subsection{Concept}
    \subsection{System Architecture}
        \paragraph{Hardware}
            \subparagraph{Controls}
            \subparagraph{User Interface}
        \paragraph{DMG-001 Mods}
            \subparagraph{Sound}
            \subparagraph{Midi Functionality}
            \subparagraph{Screen}
            \subparagraph{Power}
        \paragraph{Embedded Software}
    \subsection{Production}
        \paragraph{Resources}
        \paragraph{Schedule}

\section{Discussion}
    \subsection{Minimum Viable Product}
    \subsection{Evaluation Criteria}

\section{Conclusions}



%=================================================%
%       REMEMBER TO SORT THE BIBLIOGRAPHY
%=================================================%

\begin{thebibliography}{99}

    \bibitem[Collins et al., 2014]{COLLINS2014}
    Collins, K. and Kapralos, B. and Tessler, H. and Paul, J. L.
    \bibYear{2014}
    \bibTitle{The Oxford Handbook of Interactive Audio.}
    USA: Oxford University Press.

    
    \bibitem[Marquez, 2014]{MARQUEZ2014}
    Marquez, I.
    \bibYear{2014}
    Playing new music with old games: The chiptune subculture.
    \bibTitle{G| A| M| E Games as Art, Media, Entertainment}
    [Online], 1 (3), pp. 67-79. Available from: 
    \bibUrl{\url{http://www.gamejournal.it/wp-content/uploads/2014/04/GAME_3_Subcultures_Journal_Marquez.pdf}}
    \bibUrldate{20 October 2019}.


    \bibitem[Nintendo, 2019]{NINTENDO2019}
    Nintendo
    \bibYear{2019}
    \bibTitle{Game Boy, Game Boy Color, Game Boy Pocket Technical Data}
    [Support page] [Online] Available from: 
    \bibUrl{\url{https://www.nintendo.co.uk/Support/Game-Boy-Pocket-Color/Product-information/Technical-data/Technical-data-619585.html}}
    \bibUrldate{20 October 2019}.

\end{thebibliography}




%================================================================================


\subparagraph[]{Section 1}
    \textbf{Overall project aims and rationale Who is your project aimed at?}

    \textbf{In what situation/context is it designed to be used?}
    
        \begin{itemize}
        \item Live performance
        \item Music production
        \end{itemize}

    \textbf{How and why will people engage with it?}

        \begin{itemize}
        \item It will be an easy and straightforward way of making chiptune music
        \end{itemize}

\subparagraph[]{Section 2}

    \textbf{Details of project What are the key hardware/software elements in your
    project?}

        \begin{itemize}
            \item Sound engine: GameBoy DMG-01 (1989)
            \item New Hardware Interface
        \end{itemize}

    \textbf{What sounds will your system work with?}
    The system will generate sound using the sound chip of the GameBoy

    \textbf{What will the relationship be between user inputs and the sound 
    parameters (mapping)?} 
    To interact with the sound, the Midi protocol will be used.
    Since the GameBoy can't read and understand Midi messages,
    a translation unit is required amid the interface and the sound engine (i.e. an Arduino board).

    \textbf{How does this mapping support your overall project aims?}

\subparagraph[]{Section 3}

    \textbf{Evidence of contextual awareness, research and reading What other similar
    systems have you looked at? How has your idea developed from this research? }
    
    \textbf{What relevant concepts have fed into your design process?}

\subparagraph[]{Section 4}
    
    \textbf{Plan for implementation What resources do you require to complete your
    project? What specific tasks do you need to complete and by when?}

    \textbf{This should be written using appropriate academic language with reference
    to relevant texts/media using Harvard format.}

%    And this is my first resource: \cite{kevin}


%\printbibliography

\end{document}

