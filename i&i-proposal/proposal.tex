\documentclass[]{article}


\author{Silvio Gregorini}
\title{}
\date{2019}

% Ask if it's possible to use a two column layout
\usepackage[utf8]{inputenc}
\usepackage{csquotes}
\usepackage[english]{babel}
\usepackage[style=authoryear]{biblatex}
\addbibresource{bibliography.bib}

\begin{document}

    In this module, you will develop an original interactive or reactive system
 in which music or sound is a key component. Innovative and imaginative
  methods of interaction by a participant should be explored, and this
   could be for any relevant context (performance, composition, installation,
    game, sound toy, etc).
    
During the first half of the module, you should consider and develop a 
written proposal and plan for this project. Your proposal should be around 
2000 words and address the following areas:

\subparagraph[]{Section 1}
    \textbf{Overall project aims and rationale Who is your project aimed at?}

    \textbf{In what situation/context is it designed to be used?}
    
        \begin{itemize}
        \item Live performance
        \item Music production
        \end{itemize}

    \textbf{How and why will people engage with it?}

        \begin{itemize}
        \item It will be an easy and straightforward way of making chiptune music
        \end{itemize}

\subparagraph[]{Section 2}

    \textbf{Details of project What are the key hardware/software elements in your
    project?}

        \begin{itemize}
            \item Sound engine: GameBoy DMG-01 (1989)
            \item New Hardware Interface
        \end{itemize}

    \textbf{What sounds will your system work with?}
    The system will generate sound using the sound chip of the GameBoy

    \textbf{What will the relationship be between user inputs and the sound 
    parameters (mapping)?} 
    To interact with the sound, the Midi protocol will be used.
    Since the GameBoy can't read and understand Midi messages,
    a translation unit is required amid the interface and the sound engine (i.e. an Arduino board).

    \textbf{How does this mapping support your overall project aims?}

\subparagraph[]{Section 3}

    \textbf{Evidence of contextual awareness, research and reading What other similar
    systems have you looked at? How has your idea developed from this research? }
    
    \textbf{What relevant concepts have fed into your design process?}

\subparagraph[]{Section 4}
    
    \textbf{Plan for implementation What resources do you require to complete your
    project? What specific tasks do you need to complete and by when?}

    \textbf{This should be written using appropriate academic language with reference
    to relevant texts/media using Harvard format.}

    And this is my first resource: \cite{kevin}


\printbibliography

\end{document}

