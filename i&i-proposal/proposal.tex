\documentclass[]{article}


\author{Silvio Gregorini}
\title{}
\date{2019}

% Ask if it's possible to use a two column layout
\usepackage[utf8]{inputenc}
\usepackage{csquotes}
\usepackage[english]{babel}
%\usepackage[style=authoryear]{biblatex}
%\addbibresource{bibliography.bib}

\newcommand{\bibTitle}[1]{\emph{#1}}
\newcommand{\bibYear}[1]{(#1)}
\newcommand{\bibUrl}[1]{\textless#1\textgreater}
\newcommand{\bibUrldate}[1]{[Accessed #1]}

\newcommand{\citeInText}[2]{#1 (#2)}

% Change appearance of numeric labels in citation call-outs
\usepackage{cite}
\renewcommand\citeleft{(}
\renewcommand\citeright{)}

% Change appearance of numeric labels in bibliography 
\makeatletter
\renewcommand{\@biblabel}[1]{}
\makeatother


\begin{document}

    In this module, you will develop an original interactive or reactive system
 in which music or sound is a key component. Innovative and imaginative
  methods of interaction by a participant should be explored, and this
   could be for any relevant context (performance, composition, installation,
    game, sound toy, etc).
    
During the first half of the module, you should consider and develop a 
written proposal and plan for this project. Your proposal should be around 
2000 words and address the following areas:


%================================================================================
%PUT NUMBERS ON PARAGRAPHS

\section{Introduction} %DONE

    The aim of this paper is to propose the design and production of an 
    hardware synthesizer, starting from an existing digital sound engine.
    In the sections that follow, it will be explained the background and the rationale 
    behind the project, as well as the initial research undertaken.
    Then, it will be depicted a "blueprint" of the product: the general concept, the 
    architecture of the system -- both hardware and software -- and the production plan 
    will be covered in detail.
    In the last part the evaluation criteria will be set, in order to have a concrete measure
    of the work outcomes.

\section{Background and Motivation}

    This project has been shaped and will be realised keeping as a pivotal point the
    \emph{chiptune} subculture and its principles. As will be discussed in the next paragraphs,
    the product is conceived to be used by people already familiar with the environment and 
    limitations of this musical style. Its purpose is to give users a different and more 
    modern way to interact with a well-known set of sounds and synthesis capabilities.

    \subsection{Pushing the Limits Using Contraints} %WIP
        \paragraph{Chiptune}
        As stated by Collins et al. (2014 )\nocite{COLLINS2014} The term \emph{chiptune} refers to a style of electronic music where 
        sounds are generated using Programmable Sound Generator (PSG) sound chips 
        in vintage digital hardware, such as computers, video game consoles, arcade machines \ldots
        
        (also known as chiptunes, chip music or 8-bit music) is electronic
music that uses the microchip-based audio hardware of early home computers
and gaming consoles and repurposes it for artistic expression. Chiptune artists
reinvent the technology found in old computers such as Commodore 64,
Amiga and ZX Spectrum as well as in outdated video game consoles such as
Nintendo Game Boy or Mega Drive/Genesis in order to create new music.
The evolution of sound throughout the history of video games has been
based on the technological capabilities of the computers or game consoles
in which the game are played (McDonald, 2004). As with the visual side,
the history of video game music is highlighted by the type of technology
available at that time. As a result, we have the 8-bit, 16-bit, 64-bit, and the
128-bit eras. The first video games lacked a sound component, included only
a brief theme, a few sound effects or were limited to simple melodies by early
sound synthesizer technology.
        \paragraph{Game Boy DMG-001 and LSDJ} %name
    \subsection{Rationale} %name


\section{Name of the Project} %name
    \subsection{Concept}
    \subsection{System Architecture}
        \paragraph{Hardware}
            \subparagraph{Controls}
            \subparagraph{User Interface}
        \paragraph{DMG-001 Mods}
            \subparagraph{Sound}
            \subparagraph{Midi Functionality}
            \subparagraph{Screen}
            \subparagraph{Power}
        \paragraph{Embedded Software}
    \subsection{Production}
        \paragraph{Resources}
        \paragraph{Schedule}

\section{Discussion}
    \subsection{Minimum Viable Product}
    \subsection{Evaluation Criteria}

\section{Conclusions}



%=================================================%
%       REMEMBER TO SORT THE BIBLIOGRAPHY
%=================================================%

\begin{thebibliography}{99}

    \bibitem[Collins et al., 2014]{COLLINS2014}
    Collins, K. and Kapralos, B. and Tessler, H.
    \bibYear{2014}
    \bibTitle{The Oxford Handbook of Interactive Audio.}
    USA: Oxford University Press.

\end{thebibliography}




%================================================================================


\subparagraph[]{Section 1}
    \textbf{Overall project aims and rationale Who is your project aimed at?}

    \textbf{In what situation/context is it designed to be used?}
    
        \begin{itemize}
        \item Live performance
        \item Music production
        \end{itemize}

    \textbf{How and why will people engage with it?}

        \begin{itemize}
        \item It will be an easy and straightforward way of making chiptune music
        \end{itemize}

\subparagraph[]{Section 2}

    \textbf{Details of project What are the key hardware/software elements in your
    project?}

        \begin{itemize}
            \item Sound engine: GameBoy DMG-01 (1989)
            \item New Hardware Interface
        \end{itemize}

    \textbf{What sounds will your system work with?}
    The system will generate sound using the sound chip of the GameBoy

    \textbf{What will the relationship be between user inputs and the sound 
    parameters (mapping)?} 
    To interact with the sound, the Midi protocol will be used.
    Since the GameBoy can't read and understand Midi messages,
    a translation unit is required amid the interface and the sound engine (i.e. an Arduino board).

    \textbf{How does this mapping support your overall project aims?}

\subparagraph[]{Section 3}

    \textbf{Evidence of contextual awareness, research and reading What other similar
    systems have you looked at? How has your idea developed from this research? }
    
    \textbf{What relevant concepts have fed into your design process?}

\subparagraph[]{Section 4}
    
    \textbf{Plan for implementation What resources do you require to complete your
    project? What specific tasks do you need to complete and by when?}

    \textbf{This should be written using appropriate academic language with reference
    to relevant texts/media using Harvard format.}

%    And this is my first resource: \cite{kevin}


%\printbibliography

\end{document}

